\documentclass{article}
\XeTeXdefaultencoding utf-8
\usepackage{fontspec}
\defaultfontfeatures{Mapping=tex-text}
%\setmainfont{Linux Libertine O}
\setmainfont{Minion Pro} 
\usepackage[margin=4cm]{geometry}
\usepackage[francais]{babel}

\title{Un denier celtibèrique à Annecy\footnote{Il m’est agréable de remercier Michel Amandry, Julia Genechesi, Dominique Hollard, Fernando López Sánchez et Joël Serralongue pour leur aide et leurs conseils}}
\date{}

\begin{document}

\maketitle


Un denier celtibèrique à Annecy\footnote{l m’est agréable de remercier Michel Amandry, Julia Genechesi,
 Dominique Hollard, Fernando López Sánchez et Joël
  Serralongue pour leur aide et leurs conseils.}


En juillet 1909, lors des fouilles de la parcelle 414bis des frères 
Lacombe\footnote{Actuelles parcelles EL78, EL28 et EL90.}, située sur le forum de
 Boutae, 56 monnaies ont été mises au jour\footnote{Marteaux et Le Roux, 1913, p. 210-211.}
L’une d’elles, pourtant bien décrite, n’a, jusque là, pas pu être déterminée. 
Vraisemblablement inconnus de Charles Marteaux et Marc Le Roux, les caractères ibériques
 de ce denier n’ont pas permis son identification au début du \textsc{xx}\textsuperscript{e} siècle.
 En 1913, ils en donnent la description suivante : « Un denier
 d’argent qu’il nous a été impossible d’identifier. d/Tête d’Apollon nue à dr.
 r/ cavalier avec la lance en arrêt galopant à dr. ; au dessous XXXXXXXXXXXX . Belle
 pièce de style grec dégénéré\footnote{Marteaux et Le Roux 1913, p. 210.}. »

À l’occasion de la reprise du dossier numismatique annécien, cette description
 a été présentée à Dominique Hollard\footnote{Cabinet des Médailles, BnF.} qui l’a aisément identifiée.

Il s’agit d’un denier des Celtibères Arévaques, de Sekobirikes, probablement frappé entre les dernières années du
\textsc{ii}\textsuperscript{e} s. et les deux premières décennies du \textsc{i}\textsuperscript{er} s. av. J.-C.

~

d/ Tête nue à d. ; derrière, croissant en dessous S\footnote{En caractères ibériques.}

r/ Cavalier à d. tenant une lance, en dessous, SEKOBIRIKES\footnote{En caractères ibériques.}

Env. 19-20 mm ; 3,5-3,9 g\footnote{Feugère et Py 2011, p. 395.}

Ref : Villaronga 1994, p. 292 ; Feugère et Py 2011, SKB-292.

~

L’exemplaire annécien n’étant pas localisé, la figure 1 montre une monnaie
équivalente, conservée au Cabinet des Médailles (BnF).

[FIGURE]

\section{D’où proviennent les monnaies à légende Sekobirikes ?}

Le siège et la chute de Numance en 133, dirigés par Scipion Emilien, mettent fin aux guerres contre 
les Celtibères et assoient l’hégémonie romaine en Hispanie. Mais à la 
fin du \textsc{ii}\textsuperscript{e} s. la région n’est pas pacifiée pour autant.
L’attaque des Cimbres, les révoltes des Celtibères et des Lusitaniens\footnote{Barrandon 2007, p. 229} ont
sans doute nécessité l’intervention d’importants contingents militaires	
Mais Rome visiblement occupée par les révoltes serviles et les assauts des Teutons
et des Cimbres, n’envoie pas d’armée en Espagne\footnote{Barrandon 2007, p. 228}.
Les généraux font alors, probablement, appel, comme les accords antérieurs le prévoyaient\footnote{App., 44.}
à des troupes d’auxiliaires, recrutées et rémunérées par les cités alliées\footnote{Cadiou 2008, p. 670 ; Barrandon 2007, p. 228.
}.
Ainsi, c’est pour fournir une partie de la solde de ces guerriers indigènes qu’auraient été
frappées, dès le \textsc{ii}\textsuperscript{e} s., des monnaies à légendes ibériques, dont les exemples
sont nombreux : Kese, Iltirta, Ikalesken, Bolksan, etc. C’est aussi le cas des
deniers à l’exergue Sekobirikes. La masse des découvertes dans la haute
vallée du Duero a permis de localiser leur frappe au nord de l’actuelle province
de Soria, non loin de l’ancienne Numance\footnote{Garcia-Bellido 1974 p. 382-386 ; López Sánchez 2010, p. 176}.
C’est avec des Celtibères provenant
de ce secteur qu’eut lieu une importante campagne contre les Lusitaniens à
la fin du \textsc{ii}\textsuperscript{e} s.\footnote{App., 100.} et que la ville celtibère de Colenda (que Fernando López
Sánchez propose d’identifier à l’ancienne Numance) fut détruite en 97\footnote{App., 100.}, et
peut-être une nouvelle fois en 93\footnote{López Sánchez 2010, p. 176}.
Les hommes originaires de Sekaisa-Segeda (province de Saragosse) auraient donc été alliés de Rome, et en
particulier des proconsuls hispaniques Titus Didus puis Valerius Flaccus, lors
des campagnes en territoire arévaque, entre 98 et 92. Pour leurs actions
militaires dans cette région on frappe alors les monnaies de « ceux de Sekaisa »,
Sekobirikes\footnote{López Sánchez 2010, p. 178}.

Il pourrait être tentant de rapprocher la découverte d’une telle monnaie en
Haute-Savoie du retour des armées pompéiennes d’Espagne vers Rome à la
fin des années 70. Cependant, la présence de troupes hispaniques en Gaule,
entre la toute fin du \textsc{ii}\textsuperscript{e} s. et les années 90\footnote{Notamment avec l’armée romaine, à l’occasion de la lutte contre les Cimbres, selon López Sánchez 2010,
p.184.} pourrait aussi expliquer cette
trouvaille.

L’absence de contexte archéologique précis, ne nous permet malheureusement
pas de mieux comprendre sa découverte dans le \emph{vicus}, ni d’estimer la durée
et la période de sa circulation. La parcelle 414bis a en effet livré un lot
monétaire qui s’étend des Allobroges à Constantin\footnote{Marteaux et Le Roux 1913, p. 210-211.}.

Si de nombreux exemplaires ibériques ont été découverts sur le territoire
de l’ancienne province de Narbonnaise\footnote{Depeyrot 2006, p. 27 ; Feugère et Py 2011, p. 377-398.}, il est, à ce jour, le seul recensé en
Haute-Savoie\footnote{Selon la CAG 74.}.

\end{document}






