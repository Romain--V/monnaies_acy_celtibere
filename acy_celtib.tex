
Un denier celtibèrique à Annecy\footnote{l m’est agréable de remercier Michel Amandry, Julia Genechesi,
 Dominique Hollard, Fernando López Sánchez et Joël
  Serralongue pour leur aide et leurs conseils.}


En juillet 1909, lors des fouilles de la parcelle 414bis des frères 
Lacombe\footnote{Actuelles parcelles EL78, EL28 et EL90.}, située sur le forum de
 Boutae, 56 monnaies ont été mises au jour\footnote{Marteaux et Le Roux, 1913, p. 210-211.}
L’une d’elles, pourtant bien décrite, n’a, jusque là, pas pu être déterminée. 
Vraisemblablement inconnus de Charles Marteaux et Marc Le Roux, les caractères ibériques
 de ce denier n’ont pas permis son identification au début du \textsc{xx}\up{e} siècle.
 En 1913, ils en donnent la description suivante : « Un denier
 d’argent qu’il nous a été impossible d’identifier. d/Tête d’Apollon nue à dr.
 r/ cavalier avec la lance en arrêt galopant à dr. ; au dessous XXXXXXXXXXXX . Belle
 pièce de style grec dégénéré\footnote{Marteaux et Le Roux 1913, p. 210.}. »
 À l’occasion de la reprise du dossier numismatique annécien, cette description
 a été présentée à Dominique Hollard\footnote{Cabinet des Médailles, BnF.} qui l’a aisément identifiée.
 Il s’agit d’un denier des Celtibères Arévaques, de Sekobirikes, probablement frappé entre les dernières années du
\textsc{ii}\up{e} s. et les deux premières décennies du \textsc{i}\up{er} s. av. J.-C.


